\documentclass[12pt]{article}

\title{Insert Report Title Here}
\author{Benjamin White-Horne \\ \emph{Oxford University}}

\newcommand{\row}[1]{\texttt{#1}}

\begin{document}

\maketitle

\pagebreak

\section*{Abstract (W.I.P.)}

Change ringing is an artform which is almost exclusive to England, where people ring sets of bells
in continually evolving sequences, known as ``changes'' or ``rows''.  

More specifically, this project is concerned with the art of `composing' --- designing sequences of
rows for ringers to perform.  The art of composing predates computers by several centuries, so
composers have historically created compositions using pen, paper and copious amounts of both skill and
patience.  However, change ringing compositions are at their core sequences of permutations and are
therefore extremely ameanable to both mathematical and computer analysis.  Even the
first computers have been used to mechanise parts of composing (namely ``proving'' --- verifying
that a composition contains no repeated rows), and as the increasing speed of computers has been
exploited to allow computers to fully generate some compositions.

But brute force is still no match for the creativity of human composers, and one mostly unexplored area
of composing is software that \emph{aids} human composers by giving instant feedback and
visualisations to a composer whilst not requiring them to make drastic changes to their workflow.
This is analogous to what software like Musescore or Sibelius do for composing music.

\pagebreak

\tableofcontents

\pagebreak

\section{Introduction}

\subsection{Problems that this project is trying to solve}

(This bit is notes that I will write up later)

\begin{itemize}
    \item Suppose that I am composing a composition
    \item Composers don't always compose from start to finish \\
        $\Rightarrow$ a helpful composing program does not rely on entering a composition from
        start to finish
    \item I want to know if my composition is false as soon as possible whilst composing (I don't
        care if my comp doesn't start and end in rounds, I just want to build up any set of fragments
        of composition and I want to know if their combination is true or if not, exactly how they
        are false).
    \item I may want to check other properties about my composition whilst composing:
        \begin{itemize}
            \item Row search (How many 4-bell runs does my composition have;
                How many rows match \row{xxxx5678};
                Does my composition so far contain a specific row, e.g. Queens/\row{13572468})
        \end{itemize}
\end{itemize}

\subsection{An example composition workflow}

Here's an example of a potential composition workflow.  For the example, I'm going to be trying to
make a one-part peal of Ytterbium Surprise
Major\footnote{Blueline: https://rsw.me.uk/blueline/methods/view/Ytterbium\_Surprise\_Major}, with
a focus on music content over simplicity.

\subsubsection{Specifying a composition}

Any composition is primarily restricted by two factors:

\begin{enumerate}
    \item The number of rows that our composition is allowed to take up.  In this example, we are
        composing a peal, which has to include at least 5000 rows.  However, we want to `overrun'
        this value of 5000 by as few rows as possible because the more rows a composition has the
        longer it will take to ring.  In practice, values between 5000 and 5200 are almost ubiquitous,
        with the vast majority being between 5000 and 5100.
    \item The specific falseness of the method(s) used.  This arises from the fact that, when designing
        a composition, we add rows in
        atomic chunks (in 99\% of cases, these chunks are one lead of a single method).  Therefore,
        whilst composing, we could wind up in a situation where some rows are completely
        impossible to reach since every chunk that contains that row also contains at least one row
        already in the composition.  For some methods this restriction is worse than others --- I've
        chosen Ytterbium because its falseness is very predictable and easy to work with.
\end{enumerate}

\noindent Because of these restrictions, we are extremely likely to reach a situation where we have
to decide which rows to use and which to throw away.  So before starting to fill rows, we need to
specify what we want to include in our composition.  For the sake of this example, I will be using
the following:

\begin{enumerate}
    \item We want to include all 24 possible rows of each of the types \row{xxxx5678}, \row{xxxx8765},
        \row{xxxx6578}, \row{5678xxxx}, \row{8765xxxx}.
    \item We want to include the row known as ``Queens'' (\row{13572468}), and along the way we might
        ring other rows of the form \row{xxxx2468} which are also nice to have.
    \item We want as many runs of at least 4 bells as possible off either end of the change
        (the longer the better).  For example \row{345678xx} is preferrable to \row{xxxx4321}.
\end{enumerate}

\subsubsection{Starting the composition}

The first thing I want to do as a composer is to figure out how to get all 120 of the rows specified
in \#1 above before starting on the rest of the composition.  This is the first place that a computer
can help us (but currently can't), since we need to check that so far our plan (a) is true, and (b)
actually contains the 120 rows we want it to contain.

For example, all compositions must start and end in rounds (\row{12345678}), and if we were to
continue ringing from rounds without placing calls, we would ring a sequence of rows known as the
``plain course'' of Ytterbium.  An inspection of the plain course shows us that it contains the
following rows that we need: 6$\times$\row{xxxx5678}, 5$\times$\row{8765xxxx},
2$\times$\row{xxxx6578}.  It therefore makes sense that rotating 234 will give us more courses
containing more these desirable rows.

Experience and experimentation tells me that in order to get all 24$\times$ \row{xxxx5678},
24$\times$\row{8765xxxx} and 24$\times$\row{xxxx6578} from Ytterbium, we need to ring every lead of
the courses starting with the rows that correspond to the permutations of the form
${[(234)(56)]}^n \; \forall n \in \{0..5\}$.  To get the remaining rows, we need to ring the
courses starting with \row{1(234|342|423)8765} and \row{1(234|342|423)7865}.  This gives us 12
courses, totalling 2688 rows containing the 120 that we want.

These courses are non-negotiable --- they must exist in the final composition in order for it to
meet the above specification.  Therefore, we may only add new sections to the composition if they
are not false against these existing sections.

It's also worth noting that these 12 courses can be rung in 4 continous blocks of 3 courses (each
block stitched together by 3 bobs affecting just 234).  We do, however, still have to stitch these
blocks into one contiguous composition in such a way that the remainder is as musical as possible.

\subsubsection{Adding Queens}



\end{document}
