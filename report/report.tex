\documentclass[12pt]{article}

\title{Insert Report Title Here}
\author{Benjamin White-Horne \\ \emph{Oxford University}}

\newcommand{\row}[1]{\texttt{#1}}

\begin{document}

% Title page & T.o.C.
\maketitle
\pagebreak
\tableofcontents
\pagebreak

\section{What is Composing?}

\section{What is my Project Trying to Do?}

The purpose of the project is to build an application which \emph{helps} composers create better
change ringing compositions with as little change to their workflow as possible.  This aim is
roughly analogous to what applications like MuseScore and Sibelius do for musical composers --- the
goal is to create some form of `helpful paper' which allows the composer to experiment more freely
by providing instant feedback on everything that the composer wants to know about their composition.

Within this overall goal, here are some more specific things that I would expect from such an
application (most of these come from observations of existing programs, conversations with other
composers, and knowledge gained from experimenting with prototypes early on in this project):

\begin{itemize}
    \item \textbf{Ease of Use}: The application should be made so that anyone can install and use
        it.
    \item \textbf{Incremental}: The application should allow the user to incrementally build up
        their composition, in whatever order they want, starting wherever they want to.
    \item \textbf{Instant}: Feedback on important measures like truth, music content and length
        should always update instantly whenever the composer changes their composition.
    \item \textbf{Visual}: The application should display the composition to the user in way that
        makes it easy for the user to visually understand the composition they are working on.
    \item \textbf{View-Independent}: The application should allow the user to choose how they view
        the composition they are working on (and this should be configurable whilst midway through a
        composition).
\end{itemize}

\section{What has been done so far?}

\subsection{Composition Library}

This is a strange one, since Composition Library (shortened to `CompLib' for the rest of this
document) is not trying to be a composition editor, but simply an attempt to build a complete
central library of all the compositions that are known to ringing.  However, it does have a
composition editor in order to allow people to add new compositions to the library, and this
editor has a lot of features useful for experimenting with compositions (despite being designed for
a different purpose).  In fact, it does well enough that I think it's probably the best composing
tool out there right now, and I have made several compositions using only CompLib and paper for
assistance.  More than anything, I think the fact that the best tool right now isn't even designed
for the purpose highlights how much promise a custom-designed composing tool holds.

\subsubsection{Things CompLib's editor does well}

\begin{enumerate}
    \item It is really easy (in fact trivial) to install --- it is simply a website, and therefore
        has no software dependencies except a browser (which will be how most people download
        software anyway).
    \item It gives instant feedback on falseness (provided that the composition is a round block).
        The feedback is also quite good at telling the user exactly how and where the falseness
        occurs.
    \item You can chose how the composition is displayed (table of calls/courses, sequence of leads,
        or blue-line view).  All of these formats are also very easy to read and digest.
    \item If the composition is complete, CompLib has a really nice music breakdown available to see
        if your goals for the composition has been met.  However, this is only available for
        complete compositions and viewing it requires saving and reopening the composition which is
        not ideal.
\end{enumerate}

\subsubsection{Things CompLib's editor doesn't do well}

\begin{enumerate}
    \item CompLib's input format is very declarative --- in essence, the user enters the composition
        as a sequence of instructions for how to build the rows and these instructions are read
        in order and used to generate (`prick') the rows of your composition.  These rows are then
        used to calculate the falseness of your composition, and what will be displayed to you.
        This, however, is frustrating when you're experimenting with compositions, since a change
        near the top of a composition will cause different rows to be generated for the rest of the
        composition.  This behaviour makes perfect sense because CompLib's editor is designed for
        inputting a composition which you already know (not for experimentally building up a new
        composition) but is nonetheless makes experimentation frustrating.
    \item All of the row generation happens server-side.  On the one hand, this means that CompLib
        will run properly even on really slow machines, but also it means that any edit you make
        takes two trips across the internet before displaying the updated rows are displayed to you.
        Also, there are weird concurrency issues where some of your edits will disappear if you make
        more edits before the changes of the first one return from the server.
    \item CompLib's doesn't handle `partial' compositions well (to be clear, this is an issue
        with using a program that is designed for inputting already existing compositions ---
        this is not a complaint about CompLib as a tool) --- you can't start a composition from
        anything other than rounds, and if the composition you entered doesn't come round
        (incomplete compositions are very unlikely to come round) then CompLib will generate
        thousands of (false) rows at the end, which causes large performance issues as well as
        making it hard to see when things are actually false.  Also, multipart compositions are only
        generated in full once an entire part is finished, which means that checking falseness for
        incomplete multi-part compositions is basically impossible.
    \item There is no way of showing only part of the composition in a different format --- you have
        to switch entirely between the different views, which usually involves a lot of scrolling to
        find the bit you actually want to see.
\end{enumerate}

\subsection{Inpact}

Inpact is a program written by Alexander Holroyd to support the workflow of experimenting with
compositions.  It provides instant feedback on truth and music, but is visually quite clunky and
doesn't support `partial' compositions.

\subsubsection{Things that Inpact does well}

\begin{enumerate}
    \item It has a folding interface --- the user can fold away sections of the composition in order
        to get a higher-level overview of the composition as a whole.
    \item It doesn't require that your composition is complete before giving feedback about truth or
        music content.
    \item It has customisable music scoring --- the user decides what music they care about.
\end{enumerate}

\subsubsection{Things that Inpact doesn't do well}

\begin{enumerate}
    \item Doesn't handle `partial' compositions --- compositions have to be a single block starting
        at rounds, but don't have to come round.
    \item There isn't a visual way to see the music in the composition.
\end{enumerate}

\section{What Does My Solution Look Like?}

\end{document}
