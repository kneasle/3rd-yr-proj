\documentclass[12pt]{article}

\title{Insert Report Title Here}
\author{Benjamin White-Horne \\ \emph{Oxford University}}

\newcommand{\row}[1]{\texttt{#1}}

\begin{document}

% Title page & T.o.C.
\maketitle
\pagebreak
\tableofcontents
\pagebreak

\section{What is Composing?}

\section{What is my Project Trying to Do?}

The purpose of the project is to build an application which \emph{helps} composers create better
change ringing compositions with as little change to their workflow as possible.  This aim is
roughly analogous to what applications like MuseScore and Sibelius do for musical composers --- the
goal is to create some form of `helpful paper' which allows the composer to experiment more freely
by providing instant feedback on everything that the composer wants to know about their composition.

Within this overall goal, here are some more specific things that I would expect from such an
application (most of these come from observations of existing programs, conversations with other
composers, and knowledge gained from experimenting with prototypes early on in this project):

\begin{enumerate}
    \item \textbf{Ease of Use}: The application should be made so that anyone can install and use
        it.
    \item \textbf{Incremental}: The application should allow the user to incrementally build up
        their composition, in whatever order they want, starting wherever they want to.
    \item \textbf{Instant}: Feedback on important measures like truth, music content and length
        should always update instantly whenever the composer changes their composition.
    \item \textbf{Visual}: The application should display the composition to the user in way that
        makes it easy for the user to visually understand the composition they are working on.
    \item \textbf{View-Independent}: The application should allow the user to choose how they view
        the composition they are working on (and this should be configurable whilst midway through a
        composition).
\end{enumerate}

\section{Why Would This be Useful?}

There are several reasons why mechanical assistance would be useful for change ringing composers in
particular.

Firstly, change ringing compositions are at their core simply sequences of permuatations, and are
therefore extremely ameanable to mathematical and computerised analysis.  Also, nearly all
compositions are fundamentally designed for human consumption, putting very low upper bounds on the
potential size of the compositions which computers have to deal with (there aren't any towers with
more than 16 (ringable) bells and humans can ring about 1,500 rows per hour, so perhaps 99.99\% of
compositions have at most 16 bells and are shorter than 10,000 rows --- and at least 99\% of these
have at most 12 bells and not much over 5,000 changes).

Secondly, mathematical properties of change ringing compositions (specifically the `truth' of a
composition) are very, very important.  For the purposes of this report, a composition is `true' if
every row contained in the composition is unique, otherwise it is `false'.  Maintaining truth of a
composition is absolutely vital, since a performance of a false composition is considered invalid
and therefore is a waste of the ringers' time.  Checking proof is also extremely easy to do with a
computer (especially due to the bounds on the size of the problem) and as a result, even the
earliest computers have been used to check compositions for truth (the story goes that one of the
first programs run on the Manchester Baby was for composition proving, but this is hard to verify).
With the increasing availability of processing power computers have been used to entirely generate
compositions with minimal input from a human, but there is still something missing from computer
generated compositions.

What is currently missing, however, is a program which allows a human composer to combine the
strengths of computers and humans (computers are incredibly fast and accurate for menial grunt-work
tasks such as truth checking, whereas a skilled human composer has a good knowledge of what makes a
`good' composition and what they are trying to achieve in each specific case).  Therefore, I propose
an application which allows a human composer to get instant feedback (truth, music content, length,
etc.) about their composition \emph{whilst they are making it}.  This support for `partial'
compositions (i.e.\ compositions which are simply chunks of rows which will be rung in some
yet-to-be-defined order) is the primary motivation for building a new program rather than simply
modifying an existing one.

\section{What has been done so far?}

In this section, I will go through existing tools and review them against the goals set above.

\subsection{Composition Library}

This is a strange one, since Composition Library (shortened to `CompLib' for the rest of this
document) is not trying to be a composition editor, but simply an attempt to build a complete
central library of all the compositions that are known to ringing.  However, it does have a
composition editor in order to allow people to add new compositions to the library, and this
editor has a lot of features useful for experimenting with compositions (despite being designed for
a different purpose).  In fact, it does well enough that I think it's probably the best composing
tool out there right now, and I have made several compositions using only CompLib, paper and a
pencil for assistance.  More than anything, I think the fact that the best tool right now isn't even
designed for the purpose highlights how much promise a custom-designed composing tool holds.

% TODO: Add numbers to these

\paragraph{Ease of use:}  CompLib does really well for this.  It is really easy (in fact trivial) to
install --- it is simply a website, and therefore has no software dependencies except a browser
(which will be how most people download software anyway).  It does need an account in order to make
compositions, but this is not really an issue.

\paragraph{Incremental:}  CompLib can't handle partial compositions at all; compositions must start
(and preferrably end) in rounds.  If the composition does not come round (as is the case for almost
all unfinished compositions), then CompLib generates thousands of repeated rows on the end of your
composition in the hope that the comp will come round, to the detriment of performance and
usability.

\paragraph{Instant:}  CompLib does reasonably well.  If the composition is a round block, then it
gives a good visual indication of where the falseness occurs.  However, it isn't perfect.  The main
problem is that all the row expansion (known as `pricking') is handled server-side, which means that
any change to the composition has to travel across the internet before the user gets any feedback.
Also, there is a concurrency bug where making too many changes in quick succession will cause some
of them to disappear.

\paragraph{Visual \& View-Independence:}  CompLib again does reasonably well (and it is flawed only
because it is designed for inputting --- or later learning --- already finished compositions).
CompLib allows the user to choose between a range of several ways of viewing their composition (even
when it doesn't come round), and all of these are very easy to read.  The blueline view is, however,
missing music highlighting.  However, if you want to see more detail of one segment of a
composition, there is no way to `unfold' only part of it --- you have to switch to an entirely new
view and then find that location again.

\begin{enumerate}
    \item If the composition is complete, CompLib has a really nice music breakdown available to see
        if your goals for the composition has been met.  However, this is only available for
        complete compositions and viewing it requires saving and reopening the composition which is
        not ideal.
    \item CompLib's input format is very declarative --- in essence, the user enters the composition
        as a sequence of instructions for how to build the rows and these instructions are read
        in order and used to generate (`prick') the rows of your composition.  These rows are then
        used to calculate the falseness of your composition, and what will be displayed to you.
        This, however, is frustrating when you're experimenting with compositions, since a change
        near the top of a composition will cause different rows to be generated for the rest of the
        composition.  This behaviour makes perfect sense because CompLib's editor is designed for
        inputting a composition which you already know (not for experimentally building up a new
        composition) but is nonetheless makes experimentation frustrating.
\end{enumerate}

\subsection{Inpact}

Inpact is a program written by Alexander Holroyd to support the workflow of experimenting with
compositions.  It provides instant feedback on truth and music, but is visually quite clunky and
doesn't support `partial' compositions.

\subsubsection{Things that Inpact does well}

\begin{enumerate}
    \item It has a folding interface --- the user can fold away sections of the composition in order
        to get a higher-level overview of the composition as a whole.
    \item It doesn't require that your composition is complete before giving feedback about truth or
        music content.
    \item It has customisable music scoring --- the user decides what music they care about.
\end{enumerate}

\subsubsection{Things that Inpact doesn't do well}

\begin{enumerate}
    \item Doesn't handle `partial' compositions --- compositions have to be a single block starting
        at rounds, but don't have to come round.
    \item There isn't a visual way to see the music in the composition.
\end{enumerate}

\section{What Does My Solution Look Like?}

\end{document}
